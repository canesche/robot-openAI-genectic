\documentclass[article, a4paper, 12pt]{article}
\usepackage[brazil]{babel} %linguagem do documento
\usepackage[utf8]{inputenc} %reconhece acento e cedilha
\usepackage[top=3cm,left=2.5cm,right=2.5cm,bottom=2cm]{geometry} %margens
\usepackage{setspace}
%\usepackage{bbm.sty}
\usepackage{graphicx,url}
\thispagestyle{empty}
% coloracao
\usepackage{hyperref}
\hypersetup{
	colorlinks=true,
    urlcolor=blue
}

\begin{document}
	\onehalfspacing	
	\begin{center}
		\large{\textbf{INF628 - Estratégias de Busca em Inteligência Artificial}} \\
		\Large{Relatório Trabalho 2 \\ Algoritmo Genético}
	\end{center}
	\begin{large}
		\textbf{Aluno:} Michael Canesche \hspace{2cm} \textbf{Matrícula:} 68064 \\ \textbf{Prof.:} Levi Lelis \hspace{4cm} \textbf{Data de entrega:} 06/05/2019
	\end{large}




\section{Informações essenciais}

O código fonte pode ser encontrado no github pelo link: 

$\cdot$ \url{https://github.com/canesche/robot-openAI-genectic}

\subsection{Dependências}

A versão utilizada do python foi a 3.6.7, mas é totalmente replicável para as versões mais recentes.\\
O comando abaixo é para instalar as dependências (caso seja necessário).

$\cdot$ \textit{pip3 install --user -r requirements.txt}

\subsection{Como rodar}

Para executar o código basta utilizar o comando abaixo:

$\cdot$ \textit{python3 main.py}\\
Para executar o melhor indíviduo gerado:

$\cdot$ \textit{python3 main.py data$\backslash$individualbest.txt}

\section{Sobre o trabalho}

O objetivo do trabalho é fazer com que o robô bípede (Figura 1) aprenda a andar por um caminho que muda ligeiramente a cada interação, utilizando como algoritmo de aprendizagem o algoritmo genético. O robo é controlado pelas ações dos joelhos e as duas juntas do quadril, sendo essas ações valores que variam dentre -1 a 1, nos quais são respectivamente as velocidades angulares do joelho 1, quadril 1, joelho 2, quadril 2. Além disso, o ambiente oferece também observações feitas pelos sensores do robo, no qual não foi utilizado nesse trabalho.

\begin{figure}[!htb]
     \centering
     \includegraphics[scale=0.5]{img/robot_begin.png}
     \caption{Robô bípede no cenário}
     \label{fig1}
\end{figure}

\section{Solução proposta}

Abaixo tem um esquema de como foi modelado o problema: \\

$\cdot$ \textbf{Gene ou Ação}: vetor de tamanho 4, definido pelos valores das ações dos joelhos e dos quadris.

$\cdot$ \textbf{Individuo}: Conjunto de ações ou conjunto de genes. No trabalho foi definido um valor constante de 16 ações. Esse valor foi escolhido empiricamente devido aos resultados obtidos pela função de adaptação.

$\cdot$ \textbf{População}: Conjunto de 100 indíviduos. Esse valor foi escolhido por ser mais do que suficiente para gerar uma boa população. 

$\cdot$ \textbf{Seleção}: O algoritmo de seleção utilizado foi o torneio aleatório, no trabalho é selecionado 10 indíviduos aleatórios e dentre eles é selecionado o melhor.

$\cdot$ \textbf{Crossover}: Nessa etapa foi selecionado uma parte de tamanho aleatório e a partir desse tamanho é trocado por entre os dois indivíduos. Criando-se assim dois filhos que vieram da combinação dos pais.

$\cdot$ \textbf{Mutação}: Enfim, foi criado um grupo elite, no qual é formado pelos melhores (20 indivíduos) da atual geração. Eles possuem a probabilidade de serem gerados uma probabilidade para verificarem se serão mutados ou não em seus genes. Além disso, foi criado um hall da fama, sendo eles os melhores dos melhores das gerações e eles são sempre passados para as próximas gerações. Note que esses 5 super individuos tem a probabilidade de serem mutados, o fato aqui de usar esse artíficio é para que não percam os seus genes caso a próxima geração seja pior do que a anterior.

\section{Resultados obtidos}

\section{Conclusão}


\end{document}
